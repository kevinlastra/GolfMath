\documentclass{article}
\usepackage[utf8]{inputenc}
\usepackage{tikz}
\usetikzlibrary{shapes,arrows}
\usepackage{multicol}
\renewcommand{\contentsname}{{\Huge Table de Matières}}
\usepackage{titlesec}
\titleformat{\section}{\huge\bfseries}{\thesection}{1em}{}
\titleformat{\subsection}{\LARGE\bfseries}{\thesubsection}{1em}{}
\begin{document}
%first page
\begin{center}
\pagenumbering{gobble}
\linespread{2.0}\selectfont

{\Huge U}{\huge NIVERSITÉ DE }{\Huge M}{\huge ONTPELLIER}\\
{\huge L2}{\LARGE ~INFORMATIQUE }
\\~\\~\\~\\~\\
{\Large\textbf{GOLF MATHÉMATIQUE}}
\\~\\~\\~\\~\\

\linespread{1}\selectfont

RAPPORT DE PROJET T.E.R\\
PROJET INFORMATIQUE HLIN405\\
\vfill
\end{center}
\begin{multicols}{2}
\begin{flushleft}
\textbf{Etudiants:}\\
~~M. Mike Germain\\
~~M. Benjamin Baska\\
~~M. Kevin Lastra
\end{flushleft}
\columnbreak
\begin{flushright}
\textbf{Encadrante:}\\
~~Mm. Annie Chateau
\end{flushright}
\end{multicols}
\newpage
%bilan
\pagenumbering{arabic}
\setcounter{page}{1}
\tableofcontents
\newpage
%introduction
{\textbf{\Huge Introduction}}\\~\\~\\
~~Sous la direction de Mm. Annie Chateau, notre groupe composé par Mike Germain, Benjamin Baska et Kevin Lastra à travaillé dans le développement du jeux "Golf Mathématique" comme projet du module HLIN405.

\newpage
%cahier de charges
\section{Organisation du projet}
\subsection{Objectifs et cahier de charges}
~\\~\\
wtf
\\~\\
\textbf{\large Base et Règles}
\\~\\
aucune idée 
\\~\\
\textbf{\large Interface Graphique}
\\~\\
pire
\\~\\
\textbf{\large Génération automatique de la carte}
\\~\\
x2
\\~\\
\textbf{\large Intelligence Artificielle}
\\~\\
au secours
\\~\\
\newpage
\subsection{Division du travaille}
\newpage
\subsection{Outils de travaille}
\textbf{\large Langage de programmation}\\
Le langage qu'on à choisi pour le développement du jeux, est le C++ pour 2 raisons principales:

\begin{enumerate}
\item Ce langage est un langage de "programmation orienté aux objets" (POO).
\item Grace aux modules de HLIN202 et HLIN302 on a une base de connaissance avec la quelle on peut travailler de manière très confortable.
\end{enumerate}
\textbf{\large Software de programmation graphique}\\
insérer les raison pour les quel on a décider d'utiliser QT XD\\~\\
\textbf{\large Travaille collaboratif}\\
Nous avons utiliser multiple programme:
\begin{enumerate}
\item GitHub. Ce logiciel nous permet partages les avance du travaille réaliser par chaque-un depuis diffèrent ordinateur et aussi sauvegarde des version(???).
\item Discord. Ce logiciel nous permet le partage d'écran, communication orale et écrit, les quelles sont très utile pour le développement du jeux.
\end{enumerate}
\textbf{\large Éditeur de text}\\
La production du projet est grâce a multiple éditeur de text, différence par:
\begin{enumerate}
\item Éditeur de code -
Emacs, sublime et QTCreator.
\item Éditeur \LaTeX{} - TexMaker
\end{enumerate} 
\newpage
\section{Conception}
De la premier réunion, utilisant l'image proportionnée dans le sujet du projet, on à travaillé dans une architecture, pour le quel ce jeux s'adapterait mieux.
La premier chose qu'on à définit est la structure du terrain de jeux, Terrain de NxM Node, après le Terrain générait on a structuré une classe qui manipulerait tout les entre sortie ("GameMaster") et finalement les joueur avec une classe "PlayerController".
% Define block styles
\tikzstyle{decision} = [diamond, draw, fill=blue!20, 
    text width=4.5em, text badly centered, node distance=3cm, inner sep=0pt]
\tikzstyle{block} = [rectangle, draw, fill=blue!20, 
    text width=5em, text centered, rounded corners, minimum height=4em]
\tikzstyle{line} = [draw, -latex']
\tikzstyle{cloud} = [draw, ellipse,fill=red!20, node distance=3cm,
    minimum height=2em]
    
\begin{tikzpicture}[node distance = 2cm, auto]
    % Place nodes
    \node [block] (OP) {Out Put};
    \node [block, below of=OP] (GM) {Game Master};
    \node [block, right of=GM, yshift=0cm, xshift=2cm] (Ter) {Terrain};
    \node [block, right of=Ter, yshift=0cm, xshift=2cm] (init) {Node};
    \node [block, below of=GM] (PC) {Player Controller};
    \node [block, below of=init] (IA) {Intelligence Artificial};
    \node [cloud, right of=PC, yshift=-0.5cm, xshift=1cm] (A) {if IA};
    % Draw edges
    \path [line] (init) -- node[near start]{1} node[near end]{[*..,*.]}(Ter);
    \path [line] (Ter) -- (GM);
    \path [line] (GM) -- (OP);
    \path [line] (PC) -- (GM);
    \path [line,dashed] (PC) -- (A);
    \draw [line,dashed] (A) -- (IA);
    \draw [line,dashed] (IA) -- node[above]{send request}(PC);
    \draw [line,dashed] (A) -- node[near start]{else} ++ (-2,0) --++(0,4) -- (OP);
    \draw [line,dashed] (OP) --++ (2.5,-1.2) --++ (0,-1.8) --++ (1,-0.5);
\end{tikzpicture}


\newpage
\section{Bibliographie}
\section{Annexes}
\end{document}